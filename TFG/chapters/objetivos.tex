\chapter{Objetivos y Metodología del Trabajo}\label{objetivos}

En los últimos años el uso de plataformas web educativas ha ido incrementándose. La pandemia ha sido otro factor por el cual algunas clases anteriormente presenciales ahora son \textit{online}. Por consiguiente el uso de estas plataformas es más demandado. En el área robótica esto no es una excepción, y se está haciendo especial hincapié en el desarrollo de plataformas \textit{online} que permitan programar robots.\\

Para que la plataforma pueda mejorarse y ser más atrayente a los alumnos es necesario monitorizar su uso y un posterior análisis, donde los administradores podrán ver cómo los usuarios interactúan con la plataforma y tomar decisiones a partir de los resultados. Los administradores, por lo tanto, podrán ver por ejemplo cuánto tiempo los alumnos tardan en resolver un ejercicio y qué nota obtienen, pudiendo hacer correlaciones entre los datos obtenidos.\\

Para conseguir que Unibotics cuente con las mejoras mencionadas, se han establecido los siguientes objetivos, metodología y plan de trabajo.
\newpage

\section{Objetivos}

El objetivo principal que sigue este proyecto es la integración de un sistema de monitorización automática en una aplicación web,
en concreto en la plataforma web de robótica educativa
Unibotics. Para cumplir el objetivo principal se ha marcado diferentes subjetivos:

\begin{enumerate}
\item Capturar las diferentes interacciones de los usuarios, como puede ser el tiempo que un usuario está en la plataforma Unibotics. Esos datos recogidos son llamados sondas. Estas sondas serán capturadas en la parte cliente de Unibotics, utilizando JavaScript.
\item La grabación de las sondas en una base de datos.
\item Creación de gráficas dinámicas automáticas desde de las sondas recogidas. Se mostrarán en varias páginas web dentro de la propia plataforma. Por un lado, en una página de acceso exclusivo a los administradores. Por otro lado, los usuarios podrán acceder a otra página web donde ver un histórico de sus correspondientes puntuaciones en cada ejercicio.
\end{enumerate}

\section{Metodología}

La metodología seguida en este proyecto ha comenzado con la planificación de reuniones semanales con los tutores. En estas reuniones semanales se hace un repaso de lo avanzado durante toda la semana y se fijan los nuevos objetivos para la semana siguiente. Estas reuniones permitían resolver las dudas más inmediatas. Gracias a la plataforma Slack\footnote{https://slack.com/} se ha podido mantener la comunicación a lo largo de la semana y poder resolver dudas de una manera más dinámica. En el grupo creado en esta plataforma también se encontraban los administradores y otros desarrolladores de Unibotics.\\

Para hacer el seguimiento del trabajo se ha creado un blog\footnote{https://roboticslaburjc.github.io/2021-tfg-claudia-alvarez/}, donde periódicamente se han ido escribiendo los avances realizados. El blog ha sido creado con \textit{Github-Pages}\footnote{https://pages.github.com/}\\

Para comenzar el TFG se ha realizado un primer estudio del código fuente de Unibotics para entender su funcionamiento y poder comprender dónde habría que añadir el código para la recogida de las sondas. Cuando se ha fijado un objetivo semanal en el que hubiera que añadir código nuevo, lo primero se creaba una incidencia o \textit{issue} en el repositorio de Github correspondiente de Unibotics describiendo el problema, después se creaba una nueva rama o \textit{branch} en la que se trabajaba en el código que solucionaba la incidencia para no modificar directamente el código fuente original de Unibotics. Una vez conseguido solucionar la incidencia se crea un parche o \textit{pull request} con la solución donde los administradores de la plataforma lo revisan y si está todo bien fusionan la rama creada con la original.\\

Unibotics cuenta con tres despliegues: D1 se despliega en local para los desarrolladores, D2 en test para probar la plataforma antes de producción y D3 en producción, en la nube de Amazon. En este proyecto se ha trabajado principalmente en el despliegue D1.\\

Se ha seguido el modelo de desarrollo software en espiral basado en iteraciones. En cada iteración primero se establece un objetivo, la siguiente etapa es el diseño, luego la implementación y por último, la realización de pruebas.


\section{Plan de Trabajo}
El plan de trabajo que se ha diseñado consta de varias etapas:

\begin{enumerate}
\item \textbf{Estudio de Unibotics}: Estudio del código fuente de Unibotics.
\item \textbf{Estudio de los componentes de terceros y herramientas utilizadas}: Estudio de los diferentes componentes utilizados empezando por Django. Después se estudió la base de datos Elasticsearch y el visualizador Dash y por último la herramienta de contenedores \textit{docker} .\newpage
\item \textbf{Recogida y captura de sondas}: Con Elasticsearch se graban-almacenan las diferentes sondas para la monitorización de la plataforma. Además, crear una base de datos de prueba para poder trabajar en local. Incorporar las sondas que permiten recoger las principales interacciones de los usuarios con la plataforma web. Primero con un prototipo y luego sobre Unibotics.
\item \textbf{Visualización automática de las sondas}: Una vez definido qué sondas se desean capturar y añadido el código para su recogida, se crea la aplicación web para visualizarlas y poder hacer el análisis masivo.
\item \textbf{Redacción de la memoria}: Por último, se ha escrito esta memoria utilizando Latex\footnote{https://www.latex-project.org/}.
\end{enumerate}








