\newpage
\thispagestyle{plain}			% Supress header 
\setlength{\parskip}{0pt plus 1.0pt}
\section*{Resumen}
Conocer las interacciones de los usuarios en una plataforma Web se ha convertido en un aspecto fundamental para poder analizar el comportamiento de éstos para mejorar los servicios ofrecidos por la plataforma. La monitorización y posterior análisis de ésta información permite lograr los objetivos deseados, mejorar la experiencia del usuario e incluso hacer una optimización de nuestra plataforma web. Para realizar estas tareas es necesario recoger el máximo de datos posibles para poder hacer un mejor análisis.
\\
 \\
Este Trabajo de Fin de Grado (TFG) muestra el proceso seguido para la recogida de la información de los usuarios en la plataforma web educativa Unibotics (realizando un monitoreo automático) y cómo se han visualizado estos datos para poder analizarlos. Unibotics está dirigido a estudiantes universitarios donde aprenden sobre robótica y a programar en Python a través de la realización de varios ejercicios. Estos datos recogidos permitirán saber el comportamiento de los estudiantes a los administradores y a la vez, los estudiantes podrán ver sus progresos en cada ejercicio.
\\
\\
Para la realización de esta tarea se han combinado diferentes tecnologías que han permitido tanto el almacenamiento de la información como su posterior visualización. En éste TFG la base de datos utilizada ha sido Elasticsearch y se ha elegido Dash como \textit{framework} para la creación de las visualizaciones. Elasticsearch forma parte del ELK Stack, donde tiene su propio visualizador, Kibana, pero se decidió utilizar Dash debido a su facilidad de uso, configuración y administración.

