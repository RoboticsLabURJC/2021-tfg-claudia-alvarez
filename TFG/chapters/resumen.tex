\newpage
\thispagestyle{plain}			% Supress header 
\setlength{\parskip}{0pt plus 1.0pt}
\section*{Resumen}
Conocer las interacciones de los usuarios en una plataforma Web se ha convertido en un aspecto fundamental para poder analizar el comportamiento de éstos con idea de mejorar los servicios ofrecidos por la plataforma. La monitorización y posterior análisis de esta información, mejorar la experiencia del usuario e incluso hacer una optimización de la plataforma web. Para realizar estas tareas es necesario recoger el máximo de datos posibles lo cual permite hacer un mejor análisis.
\\
 \\
Este Trabajo de Fin de Grado (TFG) muestra el proceso seguido para la \textit{recogida de la información} de los usuarios en la plataforma web educativa Unibotics (realizando una monitorización automático) y cómo se han \textit{visualizado estos datos masivos} para poder analizarlos. Unibotics está dirigido a estudiantes universitarios donde aprenden sobre robótica y a programar en Python a través de la realización de varios ejercicios. Estos datos recogidos permitirán a los administradores conocer mejor el comportamiento de los estudiantes y a la vez, los estudiantes podrán ver sus progresos en cada ejercicio.
\\
\\
Para la realización de este TFG se han combinado diferentes tecnologías que han permitido tanto el almacenamiento de la información como su posterior visualización automática. La base de datos utilizada ha sido Elasticsearch y se ha elegido Dash como entorno para la creación de las visualizaciones automáticas. Elasticsearch forma parte del ELK Stack, que tiene su propio visualizador, Kibana, pero se decidió utilizar Dash debido a su facilidad de uso, configuración y administración.

