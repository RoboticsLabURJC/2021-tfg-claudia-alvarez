\newpage
\thispagestyle{plain}			% Supress header 
\setlength{\parskip}{0pt plus 1.0pt}
\section*{Resumen}

La monitorización y su posterior análisis está siendo una parte importante a la hora de montar una página web, ya que saber las interacciones que el usuario puede tener permite conocer el comportamiento de estos y así poder lograr los objetivos deseados, mejorar la experiencia del usuario e incluso hacer una optimización de nuestra página web. Para realizar estas tareas es necesario recoger el máximo de datos posibles para poder hacer un mejor análisis.
\\
 \\
Este Trabajo de Fin de Grado (TFG) muestra cómo se han recogido datos de la página web educativa \textit{Unibotics} (realizando un monitoreo automático) y cómo se han visualizado estos datos para poder analizarlos. \textit{Unibotics} está dirigido a estudiantes universitarios donde aprenden sobre robótica y a programar en \textit{Python }a través de la realización de varios ejercicios. Estos datos recogidos permitirán saber el comportamiento de los estudiantes a los administradores y a la vez, los estudiantes podrán ver sus progresos en cada ejercicio.
\\
\\
Para poder llevar a cabo el monitoreo se ha utilizado la base de datos \textit{Elasticsearch}, el cual almacena los datos de forma de índices para después ser visualizados en una aplicación \textit{Dash} por medio de gráficas. \textit{Elasticsearch} forma parte del ELK Stack, donde tiene su propio visualizador, \textit{Kibana}, pero se decidió utilizar \textit{Dash} debido a su facilidad de uso, configuración y administración.

