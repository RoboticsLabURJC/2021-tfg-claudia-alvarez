\chapter{Conclusiones y trabajos futuros}\label{conclusión}

	En este último capítulo se detallan las conclusiones alcanzadas, así como las competencias adquiridas al realizar este TFG y futuros trabajos.
	
	\section{Conclusiones finales} 
	\label{sec:conclusiones_finales} 
El objetivo principal de este TFG se ha cumplido, ya que se ha conseguido integrar con éxito analíticas automáticas en Unibotics. Desde la propia herramienta se detecta el tipo de usuario que accede, de tal manera que es posible controlar y mostrar una visualización u otra dependiendo de este tipo de usuario. Analizando los subjetivos, mencionados en el capítulo 2, llegamos a las siguientes conclusiones:\\

\begin{itemize}
\item  El subjetivo 1 y el subjetivo 2 se han conseguido con la utilización de la herramienta de Elasticsearch. Se ha podido capturar y guardar las interacciones de los usuarios en la plataforma. Además, se ha creado un nuevo botón en los ejercicios para poder recoger las sondas relativas a la puntuación de la eficacia del código. A parte de la sonda mencionada anteriormente, se han recogido las sondas relativas al inicio y fin de sesión, de los ejercicios y de la puntuación de estilo.
\newpage
\item El subjetivo 3 se ha logrado con la integración de las visualizaciones de la información en el servidor web, gracias al entorno Dash. Ha permitido de una manera sencilla y potente la visualización de las sondas recogidas con Elasticserach. Estas visualizaciones disponen de filtros que permiten un análisis.
\end{itemize}

	\section{Competencias adquiridas} 
	\label{sec:competencias_adquiridas} 
	Durante la realización del TFG he adquirido las siguientes competencias:
		
		\begin{itemize}
			\item Ampliado mis conocimientos sobre las tecnologías web, tanto el entorno de Django como HTML, CSS y JavaScript.
			
			\item Conocer cómo funciona una base de datos en un proyecto real, en el caso de MySQL y saber desplegar e integrar una nueva base de datos, Elasticsearch. Importancia de la información para monitorizar un servicio web. 
			
			\item Comprender las tecnologías de visualizaciones automáticas de la información gracias a Dash, la cual es una herramienta rápida y eficiente.
			
			\item Aprender a utilizar GitHub como repositorio donde desarrollar proyectos en equipo, haciendo uso de incidencias y parches. 
			
			\item Trabajar en una plataforma que está en continuo desarrollo, donde se han creado nuevas funcionalidades. Trabajar en equipo con otros desarrolladores de áreas diferentes a la trabajada en este proyecto.			
		\end{itemize}
	\section{Trabajos futuros} 
	\label{sec:trabajos_futuros} 

		En esta sección se proponen futuras líneas de trabajo para poder mejorar las analíticas automáticas:

		\begin{itemize}
			\item Enriquecer las sondas que se encuentran en Unibotics añadiendo nuevas o añadiendo nuevos campos a los índices creados para recabar más información sobre el uso de la plataforma. Actualmente las sondas de puntuación de estilo y de eficacia solo se encuentran en cuatro ejercicios, así que se podrían añadir a los nuevos ejercicios que sean creados. Para recoger nuevas sondas será necesario la creación de nuevos índices en Elasticsearch y nuevas visualizaciones automáticas en la aplicación de Dash.\\
\item Incorporar a Elasticsearch medidas de rendimiento en el servidor web. El servidor está desplegado en \textit{Amazon web services} en producción por lo que se pueden añadir métricas de memoria ocupada o CPU consumida entre otras, luego se podrá correlar esas métricas con las sondas actuales tales como el número de usuarios activos en la plataforma.

\item En este TFG se ha visualizado las sondas con los datos directamente recogidos, el siguiente paso que se podría hacer es correlacionar las sondas recogidas. Es decir, por ejemplo, saber si el tiempo en el que realizan un ejercicio afecta la nota obtenida en el ejercicio. Esto nos permite hacer análisis estadísticos o investigaciones científicas, como el efecto de la \textit{gamificación}.\\
		\end{itemize}