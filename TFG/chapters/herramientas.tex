\chapter{Herramientas utilizadas}
\label{herramientas}
En este capítulo se va a describir las tecnologías web que se han utilizado en este proyecto. Para desarrollar Unibotics se ha utilizado como tecnologías del lado del cliente HTML para dar estructura, CSS para el diseño y JavaScript para definir las acciones, las tecnologías del lado del servidor han sido el \textit{framework} de Django y base de datos SQL\footnote{Structured Query Language}. Se añadió una nueva base de datos llamada Elasticsearch y como visualización de estadísticas automáticas se ha utilizado Dash.
\section{Tecnologías web}
En esta sección se hablará de las diferentes tecnologías que forman Unibotics, la plataforma web educativa donde se realiza este trabajo y las cuales han sido usadas para llevar acabo los objetivos de este.
\subsection{HTML}
HTML son las siglas de \textit{HyperText Markup Language} donde "HiperTexto" se refiere a un texto donde hay enlaces a otra página web o en la misma página permitiendo que los documentos estén interconectados entre sí, esto es una parte fundamental de la web. Con marcado hace referencia a que HTML define la estructura del documento por ejemplo que parte del documento va a ser un titulo y donde se va a encontrar y por último con lenguaje como ya hemos dicho anteriormente HTML es un lenguaje pero no de programación si no marcado.\\

La estructura de un documento HTML esta compuesto por la definición del tipo de documento con \textit{\textless!DOCTYPE html\textgreater}, el elemento  \textit{\textless html \textgreater} y \textit{\textless/html \textgreater} para dar comienzo y final al documento HTML, el elemento {\textless head \textgreater} y {\textless/head \textgreater} donde se introducen los metadatos como es el idioma del documento o el titulo que aparece en la pestaña de la página y el elemento {\textless body \textgreater} y {\textless/body \textgreater} donde se escribe todo el contenido que se quiere mostrar a los usuarios.\cite{html}

\begin{figure}[H]
    \centering
    \includegraphics[width=9cm, keepaspectratio]{img/html.jpg}
    \caption{Estructura HTML}
    \label{fig:html}
\end{figure}

Un elemento de HTML esta compuesto por etiquetas que son palabras que marcan el inicio y final de una sección. La etiqueta de apertura esta formada por una palabra o letra rodeada por '\textless' y '\textgreater' dando comienzo al elemento y normalmente con una etiqueta de cierre al igual que la de la apertura pero rodeada por '\textless/' y '\textgreater'. La etiqueta no distingue entre mayúsculas y minúsculas. Aunque haya una gran cantidad de etiquetas a veces se necesita información adicional para completar los elementos, esto se consigue gracias a los atributos. El atributo se encuentra dentro de la etiqueta de apertura con un espacio en blanco del nombre de la etiqueta o de otro atributo, el atributo esta compuesto por un nombre seguido del signo igual "=" y el valor del atributo entre comillas. Cada etiqueta tiene unos atributos asociados y estos a la vez unos valores predefinidos si se da un valor erróneo a un atributo al renderizar la página esta lo ignorará. Algunos atributos son obligatorios como en el caso de las imágenes, vídeos o enlaces como se muestra en la Figura 3.2, la etiqueta para los enlaces es \textit{a} y es obligatorio que le siga el atributo \textit{href}  para poder añadir la dirección a la que va a dirigir dicho enlace. Entre las etiquetas de apertura y cierre nos encontramos con el texto que será el contenido de la sección.\cite{etiqueta}


\begin{figure}[H]
    \centering
    \includegraphics[width=12cm, keepaspectratio]{img/elemento.png}
    \caption{Estructura de un elemento HTML}
    \label{fig:elemento}
\end{figure}

Hay dos tipos de elementos, los elementos de bloques que son los elementos que ocupan toda una linea en el documento estos son los encabezados, listas o párrafos y los elementos en linea que solo ocupan el espacio de su contenido como los botones, enlaces o imágenes. Esto se puede cambiar gracias a atributos. Para estructurar el documento disponemos de dos etiquetas generales, \textit{\textless div \textgreater} la cual crea una sección de tipo bloque y \textit{\textless span \textgreater} para crear una sección en linea.\cite{juan2}

La última versión es la HTML5 la cual es utilizada en este proyecto, como mejoras respecto a anteriores versiones está la introducción de las etiquetas de audio \textit{\textless audio \textgreater} y vídeo \textit{\textless audio\textgreater}, anteriormente la web no estaba pensado para multimedia y había que meter parches como\textit{ Flash} de \textit{Adobe}. Se introduce SVG\footnote{Scalable Vector Graphics }para hacer gráficos vectoriales y así no se podrán ver los píxeles de las imágenes, también como novedad se tiene Canvas la cual es una web de diseño gráfico y composición de imágenes y WebGL que es una especie de Canvas pero en 3D. En HTML5 si das permiso incorpora una API\footnote{Application Programming Interfaces} de geolocalización donde se puede ver la ubicación y por último tambien incorpora lo denominado \textit{Drag and Drop} que consiste en arrastrar y soltar para facilitar la interacción con el usuario.\\

\newpage
\subsection{CSS}
CSS es el lenguaje de estilo encarga del diseño y presentación de los documentos HTML. Para llamar la atención de los usuarios en páginas web es importante añadir estilo a los documentos por eso se utiliza CSS, el cual puede definirse como un atributo de HTML llamado\textit{ style}, otra forma de meter estilo es utilizando la etiqueta {\textless style\textgreater} en la cabeza del documento HTML pero la mejor opción para añadir estilo es separándolo de la estructura, es decir, del documento HTML creando una hoja de estilo \textit{.css}  debido a que es más sencillo realizar cambios y se podrá diversificar el trabajo en estructura y estilo siendo más productivo. Para vincular la hoja de estilo con el HTML se utiliza la etiqueta \textit{\textless link\textgreater} en la cabecera.\\

La estructura de una regla CSS se divide en selectores que contienen pares de propiedad-valor como se muestra en la figura 2.3. En los selectores se pone el nombre de las etiquetas las cuales quieres cambiar su estilo como puede ser la etiqueta \textit{body}, además estos selectores pueden ser el valor del atributo \textit{id} el cual es un identificador único para un elemento de HTML en este caso se pone el símbolo # antes del valor de su \textit{id}, a parte de identificar un elemento con un \textit{id} se puede utilizar el atributo \textit{class }que es un identificador para varios elementos, en este caso para añadirle estilo a los elementos pertenecientes a la misma clase al selector se le añade un punto delante.

\begin{figure}[H]
    \centering
    \includegraphics[width=12cm, keepaspectratio]{img/css.png}
    \caption{Sintaxis CSS}
    \label{fig:css}
\end{figure}
\newpage
Las propiedades indican cual es el estilo que se quiere cambiar en un elemento como puede ser el color o el tamaño. Cada propiedad tiene unos valores asociados que en algunas ocasiones sse pueden escribir de diferente manera como en el caso de los colores, se puede escribir directamente como \textit{red} o ponerlo en su valor hexadecimal o en valores RGB\footnote{ Red, Green, Blue,}.\\

A la hora de utilizar estilos se puede poner estilos contradictorios en este caso el último estilo definido será el que se acabe aplicando, esta sería la parte de cascada que indica las siglas CSS. Si se ha utilizado una hoja de estilo pero además se ha definido en una etiqueta HTML, el definido en la etiqueta será el utilizado al renderizar el documento. La herencia también es un concepto importante en CSS ya que si un elemento no tiene estilo pero esta contenido en otro elemento que si tiene, este heredará su estilo. Los identificadores únicos tendrán preferencia a añadir el estilo que los identificadores de clase, el nombre de la etiqueta como selector es el de menor preferencia entre los selectores.\cite{juan3}


\subsection{JavaScript}
JavaScript es un lenguaje de programación interpretado que permite la ejecución de código orientado a eventos.Pueden actuar sobre el navegador a través de objetos integrados como un botón. El DOM\footnote{Document Object Model}, API que representa al documento y define la manera de interactuar con él, puede ser modificado dinámicamente gracias a JavaScript. Es importante la colocación del código JavaScript ya que se ejecuta ordenadamente de arriba a abajo. \\

JavaScript puede estar en el lado del cliente haciendo que su código se ejecute en el navegador donde podrá interactuar con este, además podrá intercatuar con el documento HTML o dibujar en la página. En el lado del servidor también se puede encontrar código JavaScript pero en este caso se ejecutará en el servidor obteniendo acceso a todos los recursos, el resultado de la ejecución se manda al navegador que se lo mostrará al usuario final. Como lenguaje JavaScript en el lado servidor se tiene por ejemplo el entorno Node.js\footnote{https://nodejs.org/}.\cite{juan4}\\

Hay diferentes formas de agregar código JavaScript a un documento HTML, como ocurría con CSS la mejor opción es tener un fichero.js separado de la estructura (HTML) y el estilo (CSS) para poder trabajar de una mejor manera, para añadirlo en la cabecera del HTML se le inserta la etiqueta {\textless script\textgreater} con el atributo \textit{src} para indicar la ubicación del fichero. Para que no haya ningún problema conviene ejecutar el código cuando se haya cargado la página para ellos se utiliza el atributo \textit{onload} que indica que se ha cargado la página y se llama a una función principal del fichero JavaScript. Otras formas de introducir JavaScript en el documento HTML es directamente en sus etiquetas por ejemplo cuando ocurre un evento o  en el contenido de la etiqueta {\textless script\textgreater}.\cite{js}\\

En este TFG se ha utilizado JavaScript para detectar las diferentes interacciones de los usuarios en la página web a través de eventos. También se ha hecho uso de JQuery\footnote{https://jquery.com/}, una librería de JavaScript que hace más sencilla la programación en JavaScript. Para la apicación de Dash como visualizador de datos se ha utilizado Bootstrap\footnote{https://getbootstrap.com/}, un complemeto de JavaScript para añadir diseño a las páginas web.

\subsection{SQL}
SQL\footnote{Structured Query Language} es un lenguaje de de consulta estructurado, se utiliza para definir, manipular y gestionar los datos almacenados en una base de datos relacional. SQL es un estándar reconocido en 1986 por ANSI\footnote{American National Standards Institute} y en 1987 por ISO\footnote{International Organization for Standardization}\\

Se necesita un gestor de base de datos RDBMS\footnote{Relational Database Management System}  que se encargará de interactuar con la base de datos por ejemplo MySQL o  Access SQL. Algunos de estos gestores trabajan en local y otros en un servidor remoto.\cite{rdbms}\\

\begin{figure}[H]
    \centering
    \includegraphics[width=11cm, keepaspectratio]{img/sql.png}
    \caption{Esquema funcionamiento SQL}
    \label{fig:sql}
\end{figure}
Una instrucción SQL esta formada por comandos, cláusulas, operadores y funciones . Los comandos de una sentencia SQL se dividen en cuatro tipos:
\begin{itemize}
\item DDL (\textit{Data Definition Language}): sirve para crear o modificar la estructura de una base de datos. Algunos de los camandos más importantes de este tipo son \textit{CREATE }para crear nuevas tablas o base de datos, \textit{ALTER} para modificar la tabla de la base de datos, \textit{DROP} para eliminar tablas o índices 
\item DML (\textit{Data Manipulation Language}): sirve para hacer consultas de selección y de acción a la base de datos. Por ejemplo tenemos los comandos \textit{SELECT} para extraer los datos,\textit{ INSERT} para insertar nuevos datos, UPDATE para actualizar los datos y \textit{DELETE} para eliminarlos.
\item DCL (\textit{Data Control Language}): se utiliza para proporcionar seguridad a la base de datos. El comando para otorgar permiso es \textit{GRANT} y para retirarlos es \textit{REVOKE}
\item TCL (\textit{Transactional Control Language}): su función es administrar los cambios en los datos. De este tipo de comandos se tiene \textit{COMMIT} para guardar el trabajo realizado o \textit{ROLLBACK} para deshacer las últimas modificaciones hechas después del último \textit{COMMIT}
\end{itemize}

\newpage
Las cláusulas son condiciones de modificación para poder definir los datos. Las cláusulas más importantes son \textit{FROM} para especificar la tabla, \textit{WHERE} para definir las condiciones de los registros que se desean, \textit{GROUP BY} para hacer agrupaciones específicas de registros, \textit{HAVING} para declarar la condición que debe cumplir cada grupo, \textit{ORDER BY} para que los registros sigan un orden especificado.\\

Los operadores pueden ser lógicos, por ejemplo, \textit{AND}, \textit{OR} o \textit{NOT}) o de comparación que serían las operaciones del estilo mayor que, menor que o igual que. Las funciones se utilizan con el comando \textit{SELECT} para devolver un único valor de un grupo de registros como puede ser\textit{ AVG }que te devuelve la media o \textit{COUNT }para devolver el número de registros. En la figura 2.5 se muestra un ejemplo de una sentencia SQL con con todas sus partes.\cite{sql}\\

\begin{figure}[H]
    \centering
    \includegraphics[width=9cm, keepaspectratio]{img/ejsql.png}
    \caption{Sentencia SQL}
    \label{fig:ejsql}
\end{figure}
En este TFG se ha utilizado SQL para acceder a la base de datos de usuarios de Unibotics.

\subsection{Django}
















\section{Recogida y grabación de datos}
\section{Visualización de estadísticas automática}